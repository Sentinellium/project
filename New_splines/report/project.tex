\documentclass{article}
\usepackage[russian]{babel}
\usepackage[utf8x]{inputenc}
\usepackage{amsmath}
\usepackage{graphicx}
\usepackage{exscale}
\usepackage{listings}
\usepackage{xcolor}
\lstset { %
    language=C++,
    backgroundcolor=\color{black!5}, % set backgroundcolor
    basicstyle=\footnotesize,% basic font setting
}


\title{Интерполяция сплайнами}

\begin{document}

\maketitle

\section{Обзор}
\begin{quotation}
Пусть имеются значения функции, измеренные в нескольких точках, тогда возникает задача нахождения значения функции в промежуточных точках.\\

	Такая задача называется \textbf{задачей интерполяции} и часто возникает на практике. Сами промежуточные значения находятся с помощью \textbf{интерполяционного полинома}.Существует несколько форматов, в виде которых можно представить интерполяционный полином. Например:\\  
	\textbf{Интерполяционным полиномом Ньютона}\\
	\textbf{Интерполяционным полиномом Лагранжа}\\
	\textbf{Интерполяция сплайнами}\\
	
	Первые два полинома являются самыми распространенными из-за удобства в вычислении, взятии интегралов и производных. Однако с увеличением степени полинома увеличивается и разброс его значений.\\
	\textbf{Сплайны} - это отличный от интерполяции полиномом степени (n-1) способ. Сплайны избавляются от разброса с помощью выделения (n-1) полиномов каждый из которых находится в своей сетке а = $х_0$ < $х_1$ < ... < $Х_n$ = b. Полученные полиномы связаны между собой и образуют единый непрерывный, гладкий  график функции(для степени полинома >1).
	\end{quotation}
\section{Постановка задачи}
\begin{quotation}

	Имеется набор данных - \textbf{N} пар (\textbf{$x_i$};\textbf{$y_i$}), где \textbf{$x_i$} и \textbf{$y_i$} действительные числа, а также значение \textbf{x}, для которого требуется найти значение \textbf{y}. По данным построить интерполяционный \textbf{кубический} или \textbf{линейный сплайн} для каждой \textbf{сетки} и определить значение в точке \textbf{x}.
	\newpage
\end{quotation}
\section{Решение задачи}
\begin{quotation}
	
\textbf{Для линейных сплайнов:}\\

	Предположим, что на отрезке [а, b] задана сетка а = $х_0$ < $х_1$ < < ... < $Х_n$ = b. Не будем предполагать ее равномерности и обозначим\\
$h_i$ = $x_{i+1}$ - $x_i$.\\

	Пусть задана таблица значений функции f($x_i$) =$y_i$. Соединим прямой точки ($x_i$,$y_i$), ($x_{i+1}$,$y_{i+1}$). В результате получаем функцию, которая на отрезке [$x_i$,$x_{i+1}$] задается уравнением прямой:\\
	
	\textbf{$y=y_i+\frac{y_{i+1}-y_i}{x_{i+1}-x_i}*(x-x_i)$}.\\
	
	\textbf{Для кубических сплайнов:}\\
	
	Задача состоит в нахождении кубического сплайна, обозначим его s(x) , который образован (n-1) кубическими.\\
	
	\textbf{$P_k(x)=a_k+b_k*(x-x_k)+c_k*(x-x_k)^2+d_k(x-x_k)^3$}\\

которых определен на отрезке [$x_k$,$x_{k+1}$] и которые должны удовлетворяют нижеприведенным условиям в узлах интерполирования, состоящим в прохождении графика сплайна через заданные точки и наличия у сплайна определенной гладкости.\\
	
1. Условия прохождения через заданные точки\\

\textbf{$P_k(x_k)=f(x_k)$}.  \textbf{$P_k(x_{k+1})=f(x_{k+1})$}. k=1,2,....,n-1.\\

2. Условия непрерывности первой производной во внутренних точках\\

3. Условия непрерывности второй производной во внутренних точках\\

 Кроме этого, наложим на сплайн следующие граничные условия (для однозначного определения его коэффициентов).\\

1. На левой границе отрезка \textbf{s($x_1$)=0}\\
 
2. На правой границе отрезка \textbf{s($x_n$)=0}\\
\newpage

Итак, для k=1,2,...,n-1:\\
\textbf{$a_k=f(x_k)$}\\
\textbf{$b_k$ следует определить}\\
\textbf{$c_k=\frac{3f[x_k,x_{k+1}-b_{k+1}-2*b_k]}{h_k}$}.\\
\textbf{$d_k=\frac{b_k+b_{k+1}-2f[x_k,x_{k+1}]}{h_k^2}$}\\

где\\


\textbf{$f[x_k,x_{k+1}]=\frac{f(x_{k+1}-f(x_k)}{x_{k+1}-x_k},$}\\   
$h_k=x_{k+1}-x_k$\\

является разделенной разностью первого порядка интерполируемой функци \textbf{f(x)}, которая построена по точкам \textbf{$x_k$, $x_{k+1}$}.
Из условий непрерывности второй производной сплайна во внутренних узлах отрезка интерполирования получаем, что\\
\textbf{$c_k+3d_kh_k=c_{k+1}$.}\\
 
	Подставим в полученные уравнения приведенные выше выражения для коэффициентов \textbf{$c_k$} и \textbf{$d_k$} полиномов \textbf{$P_k$(x)} через неизвестные пока коэффициенты \textbf{$b_k$}. Тогда получаем систему линейных алгебраических уравнений относительно коэффициентов \textbf{$d_k$} , т.е. относительно тангенсов углов наклона кубического сплайна во всех (как внутренних, так и граничных) узлах отрезка интерполирования.\\
	
	Преобразуем систему уравнений, вычислив коэффициенты при \textbf{$b_k$}, \textbf{$b_{k+1}$}, \textbf{$b_{k+2}$}, получим тогда\\
	
	
\textbf{$\alpha_k$ = $\frac{1}{h_k}$}\\
\textbf{$\beta_k = 2[\frac{1}{h_k}+\frac{1}{h_{k+1}}$}\\
\textbf{$\gamma_k = \frac{1}{h_{k+1}}$}\\
\textbf{$\delta_k = 3[\frac{f[x_{k+1},x_{k+2}]}{h_{k+1}}+\frac{f[x_k,x_{k+1}]}{h_k}$}\\

	Заметим, что поскольку мы разыскиваем сплайн с нулевыми наклонами в граничных точках, то \textbf{$b_1$ = 0} и \textbf{$b_n$ = 0} , но данные коэффициенты входят в первое и последнее уравнение системы. Поэтому система из (n-2)-ух уравнений относительно неизвестных коэффициентов \textbf{$b_k$} для k=2,3,...,n-1 выглядит следующим образом:\\
	
	
$$\begin{pmatrix}-\beta_1& \gamma_1&0&0&\cdots&0&0&\delta_1\\[2pt] \alpha_2&-\beta_2&\gamma_2&0&\cdots&0&0&\delta_2\\[2pt] 0&\alpha_3&-\beta_3&\gamma_3&\cdots& 0&0&\delta_3\\[2pt] \cdots& \cdots& \cdots& \cdots& \cdots& \cdots& \cdots& \cdots\\[2pt] 0&0& \cdots&\cdots&\cdots&\alpha_n&-\beta_n&\delta_n \end{pmatrix}\!.$$\\



	Матрица системы линейных уравнений является трехдиагональной, все остальные ее коэффициенты, за исключением коэффициентов, стоящих на главной и побочных диагоналях, равны нулю.\\
 	Решение системы с трехдиагональной матрицей осуществляется за число операций, пропорциональное числу неизвестных (например, методом прогонки). Решив эту систему мы находим коэффициенты $b_1$,$b_2$,...,$b_{n-1}$ и подставляя их, а также $b_1=b_n=0$ в выражения для $C_k$ и $d_k$ при к=1,2,...,n-1 находим все коэффициенты полиномов $p_k(x)$ составляющих кубический сплайн s(x).\\

\end{quotation}
\section{Реализация}
\begin{quotation}

	Проект содержит библиотеки для работы с линейными и кубическими сплайнами и файл \textbf{main.cpp} для работы с этими библиотеками.\\

	 Интерполяция линейными сплайнами \textbf{(linear\_ spline)} реализована как библиотека, содержащая класс с несколькими массивами, содержащими в себе данные(координаты и коэффициенты полинома интерполяции), конструктор, который принимает на вход количество точек, читает координаты из файла \textbf{input.txt}, вычисляет значение коэффициентов через алгебраическое определение линейного сплайна и выводит полином в консоль, а также деструктор, удаляющий динамические массивы. Класс имеет метод \textbf{value()}, который по заданной точке \textbf{x} находит значение сплайна.\\
	 
	 	  
	Класс \textbf{cubic\_ spline} похож на \textbf{linear\_ spline} и имеет тот же метод \textbf{value()}. Отличается он конструктором, в котором находятся значения \textbf{a, b, c} (нулевой, первой и второй производной в узлах соответственно) и \textbf{d}, которые выражены таким образом, что они гарантируют непрерывность первой и второй производных функции. При задании условий так же выходит, что \textbf{b} задается рекуррентным уравнением \textbf{$\alpha_i$*$b_i$+$\beta_i$*$b_{i+1}$+$\gamma_i$*$b_{i+2}$ = $\delta_i$} Где коэффициенты перед \textbf{b} находятся через условия гладкости функции. Уравнение представляется как \textbf{СЛУ} и решается \textbf{методом прогонки}. после этого через \textbf{b} находятся значения \textbf{c, d,} позволяющие однозначно опредилить сплайн для каждой сетки.Сама программа запрашивает у пользователя на вход количество точек и метод интерполяции после чего по значениям из \textbf{input.txt} (они идут в порядке $x_1$ $x_2$ ... $x_n$ $y_1$ $y_2$ ... $y_n$) находит интерполяционный многочлен и выводит его в консоль.

\end{quotation}
\end{document}
